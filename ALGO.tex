\documentclass[12pt,a4paper]{book}

\usepackage{dsalgo}
\title{\Huge \textbf{Algoritmizace}\\[0.5cm] \Large Cvičení}
\author{}
\date{2025}

\begin{document}
\maketitle
\tableofcontents

\chapter{Posloupnosti}
\begin{definition}
\emph{Posloupnost} je zobrazení z množiny přirozených čísel do libovolné množiny.
Zápis: $(a_n)$ -- posloupnost, $a_n$ -- n-tý prvek.
\end{definition}

\section{Způsoby zadávání posloupností} 
	\subsection{Prvních x členů}
		Musí být jasné pravidlo, jak vyjádřit další členy posloupnosti.
		\begin{example}
			\begin{itemize}
				\item triviální příklad $1, 2, 3, 4 ... $
				\item fibbonacci $0, 1, 1, 2, 3, 5, 8 ...$
				\item alternující $1, -1, 1, -1 ...$
				\item konečná $2, 4, 6, ..., 20$
			\end{itemize}
		\end{example}
	\subsection{Vzorec pro n-tý člen}
		Vyjádříme obecný vzorec pro n-tý prvek na základě indexu.
		\begin{example}
			\begin{itemize}
				\item $(\frac{n}{n+1})$
				\item $((-1)^nn)$
				\item $ (1 + \frac{1}{n})^n)$
			\end{itemize}
		\end{example}
	
	\subsection{Rekurentně} 
		Rekurentní zadání obsahuje zpravidla 1. člen (nebo několik prvních členů) a pravidlo, jak vytvořit další člen ze členů předcházejících.
		\begin{example}
			\begin{itemize}
				\item $(\frac{n}{n+1})$
				\item $((-1)^nn)$
				\item $ (1 + \frac{1}{n})^n)$
			\end{itemize}
		\end{example}
Speciálním případem rekurentního zadání jsou aritmetická a geometrická posloupnost.

\begin{definition}
	\emph{aritmetická posloupnost} je zadána $a_1 = a$ a $a_{n+1} = a_n + d$.
	Podobně \emph{geometrická posloupnost} je definována $a_1 = a$ a $a_{n + 1} = a_n \cdot q$
\end{definition}
		
\section{Úkoly}
\begin{enumerate}
	\item Určete vzorec pro n-tý člen následujících posloupností:
		\begin{itemize}
			\item $ 3, 7, 11, 15, 19 ...$
			\item $ 2, 6, 18, 54, 162, 486 ...$
			\item $ 2, 6, 12,20, 30 ... $
			\item $ 1, -2, 3, -4, 5, -6 ...$
			\item $
\frac{1}{1 \cdot 4}, \quad \frac{3}{4 \cdot 7}, \quad \frac{5}{7 \cdot 10}, \quad \frac{7}{10 \cdot 13}, \quad \dots$
		\end{itemize}
	\item Vypočítejte prvních 5 prvků posloupností daných vzorcem:
		\begin{itemize}
			 \item $ a_n = 5 + 3(n - 1)$
			 \item $ a_n = 2 \cdot 4^{n - 1}$
			 \item $ a_n = n^2 + 2n + 1$
			 \item $ a_n = (-2)^n \cdot n$
			 \item $a_n = \frac{n^2 + n}{2}$
		\end{itemize}
\end{enumerate}


	
	
\chapter{Vlastnosti posloupností}

\chapter{Elementární funkce}

\chapter{Pseudokód}

\chapter{O-notace}

\chapter{Třídění}


\end{document}